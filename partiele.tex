\chapter{Parti\"ele Integratie}

De productregel voor afgeleiden levert een stelling voor integralen. \\


\begin{theorem}{Parti\"ele integratie}
	Zij $u,v$ afleidbare functies, dan is
	\begin{eqnarray}
	\int u(x)v'(x)\dx  = & u(x)v(x) -  \int u'(x)v(x) \dx 
	\end{eqnarray}

	
	 
\end{theorem}

Hierbij vallen ons twee zaken direct op:

\begin{enumerate}
	\item Het eerste deel van de som bevat geen integraal meer.
	\item Het tweede deel bevat een nieuwe afgeleide en primaire functie.
\end{enumerate}

\begin{exercise}
	Hoe zou je met bovenstaande formule onderstaande integraal oplossen?
	\[ \int  xe^x\dx \]
\iftoggle{solution}{%
	Stel $u(x) = x $ en $u'(x) = 1$ en stel $v'(x) = e^x$ zodat $v(x) = e^x$. 
	Parti\"ele integratie geeft dan
	
	\[ \int xe^x\dx = xe^x -   \int e^x\dx = xe^x - e^x + C , C \in \mathbb{R}\]
}{%
	% electronic
}	


\end{exercise}

\begin{exercise}
	Los onderstaande integraal op
	\[ \int  9x^2 ln(x)\dx \]
	\iftoggle{solution}{%
		Stel $u(x) = ln(x) $ en $u'(x) = \frac{1}{x}$ en stel $v'(x) = 9x^2$ zodat $v(x) = 3x^2$. 
		Parti\"ele integratie geeft dan
		
		\[ \int 9x^2 ln(x)\dx = ln(x)3x^2 -   \int \frac{1}{x}3x^2\dx  = 3x^3ln(x) - x^3+ C , C \in \mathbb{R}\]
	}{%
		% electronic
	}	
	
\end{exercise}

Indien je de $u(x)$ en $v(x)$ verkeerd kiest kan het zijn dat je een moeilijkere integraal bekomt. Het is dus belangrijk een goede keuze te maken voor u en v.

\begin{exercise}
	Los onderstaande integraal op
	\[ \int  x sin(x)\dx \]
	\iftoggle{solution}{%
		Stel $u(x) = x $ en $u'(x) = 1$ en stel $v'(x) = sin(x)$ zodat $v(x) = -cos(x)$. 
		Parti\"ele integratie geeft dan
		
		\[ \int x sin(x)\dx = -xcos(x) +   \int cos(x)\dx  = -xcos(x)+sin(x)+ C , C \in \mathbb{R}\]
	}{%
		% electronic
	}	
	
\end{exercise}

\begin{exercise}
	Los onderstaande integraal op
	\[ \int  xe^{2x}\dx \]
	\iftoggle{solution}{%
		Stel $u(x) = x $ en $u'(x) = 1$ en stel $v'(x) = e^{2x}$ zodat $v(x) = \frac{1}{2}e^{2x}$. 
		Parti\"ele integratie geeft dan
		
		\[ \int xe^{2x}\dx = \frac{xe^{2x}}{2} -   \int \frac{1}{2}e^{2x}\dx  = 3x^3ln(x) - x^3+ C , C \in \mathbb{R}\]
		\[ \int xe^{2x}\dx = \frac{xe^{2x}}{2} - \frac{1}{4}\int e^{2x}d2x = \frac{xe^{2x}}{2} - \frac{1}{4}e^{2x}+ C , C \in \mathbb{R} \]
	}{%
		% electronic
	}	
	
\end{exercise}

Enkele tips om parti\"ele integratie te herkennen:
\begin{itemize}
	\item Parti\"ele integratie wordt gebruikt als de integrand geen onmiddellijke integratie toelaat.
	\item Parti\"ele integratie wordt gebruikt voor integratie van een produ
	ct van twee functies, op voorwaarde dat de nieuwe integraal eenvoudiger word
	t (bv. een macht doen dalen). Eventueel moet men herhaalde malen partieel integreren.
	\item Soms krijgt men na (herhaald) partieel integreren opnieuw de gevraagde integraal terug. Zie bijvoorbeeld oefening \ref{sec:partiele-herhaling}.
	\item Het is goed even op te merken dat, als er bij het oplossen van integralen een oplossingsmethode is, deze niet noodzakelijk uniek is. 
\end{itemize}

\begin{exercise}
	\label{sec:partiele-herhaling}
	Los onderstaande integraal op
	\[ \int  sin(x) e^x \dx \]
	Stel $u(x) = e^x$ en $v'(x) = sin(x)$ zodat $v(x) = -cos(x)$
	\iftoggle{solution}{%
		\begin{eqnarray}
		\int  sin(x) e^x \dx = & -e^x cos(x) + \int e^xcos(x) \\
		& -e^x cos(x) + \int e^x d sin(x) \\
		& -e^x cos(x) + e^x sin(x) 	- \int  sin(x) e^x \dx		
		\end{eqnarray}
		We komen dus in het linkerlid en rechterlid hetzelfde tegen. Brengt men deze term uit het rechterlid naar het linkerlid krijg je 
				\begin{eqnarray}
		2 \int  sin(x) e^x \dx = & e^x( sin(x) - cos(x)) \\	
		\int  sin(x) e^x \dx = & \frac{1}{2}(e^x( sin(x) - cos(x)))
		\end{eqnarray} 
	}{%
		% electronic
	}	
	
\end{exercise}

\begin{theorem}{Parti\"ele integratie voor bepaalde integralen}
	Stel $u$ en $v$ functies gedefinieerd op het interval $I$ en $[a,b]$ zit volledig in $I$ dan geldt:
	\begin{eqnarray}
		\int_a^b u(x)v'(x) = [u(x)v(x)]^b_a - \int_a^b v(x)du(x)
	\end{eqnarray} 
\end{theorem}

\begin{exercise}
	Lost onderstaande integraal op
	\[\int_1^e ln(x) x \dx \]
	Oplossing:
	\iftoggle{solution}{%
		\[ \int_1^e ln(x) d\frac{x^2}{2} = \frac{e^2}{4} + \frac{1}{4} \]
	}
	
\end{exercise}
