\chapter{Integratie door substitutie}

\section{Lineariteit van integratie}
De afgeleide van het veelvoud van een functie is het veelvoud
van de de afgeleide van die functie.
De afgeleide van de som van twee functies is de som van de afgeleide van beide functies. \\
\begin{theorem}{Lineariteit van de integratie}
	Stel dat $f$ en $g$ twee integreerbare functies zijn, $a,b \in \mathbb{R}$
	, dan is
	\begin{eqnarray}
	\int_a^b (af(x) + bg(x)) \dx = & a \int_a^b f(x)\dx + b \int_a^b g(x) \dx
	\end{eqnarray}
\end{theorem}

De kettingregel voor de afgeleide van samengestelde functies
geeft aanleiding tot volgende stelling voor integralen.

\begin{theorem}
	Stel $f$ een functie met een primitieve F zodat 
	\[ \int f(x) \dx = F(x) +C, C \in \mathbb{R}\]
	en $g$ een afleidbare functie, dan is met $g(x) =t$  en $g'(x)\dx = dt$ 
	
	\begin{eqnarray}
	\int f(g(x)).g'(x)\dx =  \int f(t) dt   = F(g(x)) + C, C \in \mathbb{R}
	\end{eqnarray}
\end{theorem}